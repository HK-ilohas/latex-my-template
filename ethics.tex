\documentclass[uplatex]{jsarticle}
\usepackage[utf8]{inputenc}
% 図を現在位置に挿入
\usepackage{here}
% 文字コード関連の設定
\usepackage{otf}
% 箇条書きのパラメータ関連の設定
\usepackage{paralist}
\usepackage{enumitem}
% マージンの設定
\usepackage[left=30mm,right=30mm,top=35mm,bottom=30mm]{geometry}
\renewcommand{\baselinestretch}{1.1}
% 画像関連
\usepackage[dvipdfmx]{graphicx}
% フォント
\usepackage{newtxtext,newtxmath}
% URL
\usepackage{url}
% SI単位
\usepackage{siunitx}
% 文字数、行数指定
\makeatletter
\def\mojiparline#1{
\newcounter{mpl}
\setcounter{mpl}{#1}
\@tempdima=\linewidth
\advance\@tempdima by-\value{mpl}zw
\addtocounter{mpl}{-1}
\divide\@tempdima by \value{mpl}
\advance\kanjiskip by\@tempdima
\advance\parindent by\@tempdima
}
\makeatother
\def\linesparpage#1{
\baselineskip=\textheight
\divide\baselineskip by #1
}
% 名前
\newcommand{\myname}{○○ ○○}
% 出席番号
\newcommand{\mynum}{○}
\begin{document}
% 文字数
\mojiparline{40}
% 行数
\linesparpage{40}
% ページ番号を消す
\pagestyle{empty}
\begin{flushleft}
\textsf{
【課題番号】 ○-○ 【クラス】 I  【出席番号】 \mynum  【氏名】 \myname
\vspace{0.2em}
\hrule width \textwidth height 0.5pt
\vspace{0.5em}
}
\end{flushleft}

% 箇条書き
\begin{description}[style=multiline,topsep=0em,font=\textmc]
    \item[(イ)]
    ああああああああああああああああああああああああああああああああああああああああああああああああああああああああああああああ

    \item[(ロ)]  
    ああああああああああああああああああああああああああああああああああああああああああああああああああああああああああああああ.

\end{description}

\end{document}